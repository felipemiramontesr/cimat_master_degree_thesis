            
\documentclass[runningheads,a4paper]{llncs}
\def\spanishoptions{mexico}
\usepackage[spanish]{babel}

\usepackage{amssymb}
\setcounter{tocdepth}{3}
\usepackage{graphicx}
\usepackage[utf8]{inputenc}
\usepackage{url}
\usepackage{listings}
\usepackage[acronyms]{glossaries}
\setacronymstyle{long-short}

\makenoidxglossaries



%Terminos de glosario
\newglossaryentry{Swim Lanes}{name={Swim Lanes},
description={es un elemento visual utilizado en los diagramas de flujo de procesos, o diagramas de flujo, que distingue visualmente reparto del trabajo y de las responsabilidades de los sub-procesos de un proceso de negocio}}

\newglossaryentry{Visa Internacional}{name={Visa Internacional},
description={es una corporación de servicios financieros multinacional estadounidense con sede en Foster City, California, Estados Unidos. Facilita las transferencias electrónicas de fondos en todo el mundo, más comúnmente a través de tarjetas de crédito de marca Visa y tarjetas de débito.  Visa proporciona a las instituciones financieras con productos de pago de marca Visa que luego utilizan para ofrecer crédito, débito, pre-pago y programas de acceso a dinero en efectivo}}

\newglossaryentry{Ministerio Aleman de Economia}{name={Ministerio Alemán de Economía},
description={es un departamento ministerial de la República Federal de Alemania. Anteriormente se la conocía como el "Ministerio de Economía". Se volvió a crear en 2005 como "Ministerio de Economía y Tecnología", después de que previamente se había fusionado con otros ministerios para formar el Ministerio Federal de Economía y Trabajo entre 2002 y 2005}}

\newglossaryentry{Banco Aleman}{name={Banco Alemán},
description={en alemán y oficialmente: Deutsche Bundesbank o simplemente Bundesbank, es el banco central de Alemania, creado en 1957 para servir como banco central de la República Federal de Alemania.}}


\newglossaryentry{Smart Cards}{name={Smart Cards},
description={es una tarjeta pequeña con circuitos integrados embebidos, están hechas generalmente de cloruro de polivinilo de plástico, pero a veces de poliésteres basados en tereftalato de polietileno, estireno acrilonitrilo butadieno o policarbonato, pueden proporcionar identificación personal, autenticación, almacenamiento de datos y procesamiento de solicitudes así como proporcionar autenticación de seguridad para el inicio de sesión único (SSO) dentro de las grandes organizaciones}}

\newglossaryentry{Hash Functions}{name={Hash Functions},
description={Una función de hash es una función que se puede utilizar para mapear los datos de tamaño arbitrario a los datos de tamaño fijo. Los valores devueltos por una función hash se llaman valores hash, códigos hash, sumas de hash, o simplemente hashes. Un uso es una estructura de datos llamada una tabla hash, ampliamente utilizado en los programas informáticos para la rápida búsqueda de datos}}

\newglossaryentry{Identificador}{name={Identificador},
description={Los identificadores son símbolos léxicos que nombran entidades}}

\newglossaryentry{Password}{name={Password},
description={es una cadena de caracteres utilizados para la autenticación del usuario para demostrar la aprobación de identidad o de acceso para obtener acceso a un recurso (ejemplo: un código de acceso es un tipo de contraseña que debe ser mantenida en secreto)}}

\newglossaryentry{UMLsec}{name={UMLsec},
description={es una extensión para el Lenguaje Unificado de Modelado (UML) para la integración de la información relacionada con la seguridad en las especificaciones UML}}

\newglossaryentry{CC}{name={Common Criteria},
description={Common Criteria para Information Technology Security Evaluation (abreviado como: Common Criteria o CC) es un estandar internacional (ISO/IEC 15408) para la certificación de la seguridad. Actualmente se encuentra en la version 3.1 revision 4}}

\newglossaryentry{KISS}{name={KISS},
description={es un acrónimo de "Keep It Simple, Stupid" como un principio de diseño señalado por la Marina de Estados Unidos en 1960. El principio KISS afirma que la mayoría de los sistemas funcionan mejor si se mantienen simples en lugar de hacerse complicados; Por lo tanto, la simplicidad debe ser un objetivo clave en el diseño y la complejidad innecesaria debe evitarse. La frase se ha asociado con el ingeniero aviones Kelly Johnson (1910 a 1990)}}

\newglossaryentry{SPARK}{name={SPARK},
description={es un lenguaje de programación especialmente diseñado para sistemas de alta integridad. Es un subconjunto anotado de Ada desarrollado por la empresa británica Praxis High Integrity Systems, que elimina ciertas características del lenguaje consideradas peligrosas en este tipo de sistemas (como las excepciones o la sobrecarga de operadores), y que añade anotaciones formales para realizar automáticamente análhampton Program Analysis Development Environment), un conjunto de herramientas destinadas al análisis de flujo de datos y de información. El nombre SPARK deriva de SPADE Ada Kernel}}

\newglossaryentry{Visual Paradigm}{name={Visual Paradigm},
description={es una herramienta CASE UML en apoyo a UML 2, SysML y Business Process Modeling Notation del Object Management Group}}

\newglossaryentry{Praxis}{name={Praxis},
description={Integrity Partnership Creativity Excellence,
The Foremost International Specialist in Critical Systems Engineering. \url{http://www.praxis-his.com}}}

\newglossaryentry{Metodos Tradicionales}{name={Métodos Tradicionales},
description={es una disciplina de trabajo sobre el proceso de desarrollo del software, con el fin de conseguir un software más eficiente. Para ello, se hace énfasis en la planificación total de todo el trabajo a realizar y una vez que está todo detallado, comienza el ciclo de desarrollo del producto software. Se centran especialmente en el control del proceso, mediante una rigurosa definición de roles, actividades, artefactos, herramientas y notaciones para el modelado y documentación detallada. Las metodologías tradicionales no se adaptan adecuadamente a los cambios, por lo que no son métodos adecuados cuando se trabaja en un entorno, donde los requisitos no pueden predecirse o bien pueden variar}}

\newglossaryentry{Metodos Agiles}{name={Métodos Ágiles},
description={es un proceso incremental, pequeño y frecuente con entregas con ciclos rápidos, también Cooperativo (clientes y desarrolladores trabajan constantemente con una comunicación muy fina y constante), sencillo (El método es fácil de aprender y modificar para el equipo, es documentado y adaptativo (capaz de permitir cambios de último momento). Las metodologías ágiles proporcionan una serie de pautas y principios junto a técnicas pragmáticas que puede  que  no  curen  todos  los  males  pero  harán  la  entrega  del  proyecto  menos  complicada y más satisfactoria tanto para los clientes como para los  equipos de entrega}}

\newglossaryentry{Taxonomia}{name={Taxonomia},
description={es la práctica y ciencia de la clasificación. La palabra también se utiliza como sustantivo, una taxonomía o esquema taxonómico, es una clasificación particular}}

\newglossaryentry{Software}{name={Software},
description={es cualquier conjunto de instrucciones que dirige un sistema o equipo electrónico para llevar a cabo operaciones específicas. Las aplicaciones informáticas se compone de programas de ordenador, bibliotecas y datos no ejecutables relacionados (como la documentación en línea o los medios digitales)}}

\newglossaryentry{Cross-Site Scripting}{name={Cross-Site Scripting},
description={es un tipo de inseguridad informática o agujero de seguridad típico de las aplicaciones en red también conocido como XSS, que permite a una tercera persona inyectar en páginas web visitadas por el usuario código JavaScript o en otro lenguaje similar (ej: VBScript), evitando medidas de control como la Política del mismo origen}}

\newglossaryentry{Pruebas Fuzz}{name={Pruebas Fuzz},
description={es una técnica sencilla para la alimentación de entrada al azar a las aplicaciones}}

\newglossaryentry{Framework}{name={Framework},
description={es una abstracción en la que el software que proporciona una funcionalidad genérica se puede cambiar de forma selectiva por el código escrito por el usuario adicional, proporcionando así el software de aplicación específica}}

\newglossaryentry{Criptografia}{name={Criptografía},
description={es una disciplina científica enfocada a crear fórmulas matemáticas que proporcionan altos niveles de seguridad. Dichas fórmulas son aplicadas a mensajes para hacer representaciones lingüísticas ininteligibles a simple vista y que se puedan descifrar únicamente conociendo la fórmula o las llaves que se usaron para cifrarlo}}

\newglossaryentry{Online}{name={Online},
description={es un estado de conectividad, frente al término fuera de línea (Offline) que indica un estado de desconexión}}

\newglossaryentry{Open Source}{name={Open Source},
description={se refiere a un programa o software en el que el código fuente (la forma del programa cuando un programador escribe un programa en un lenguaje de programación en particular) está a disposición del público en general para su uso y / o modificación de su diseño original sin cargo . Código fuente abierto se crea normalmente como un esfuerzo de colaboración en el que los programadores mejorar el código y comparten los cambios dentro de la comunidad}}

\newglossaryentry{Hashing}{name={Hashing},
description={es la producción de valores hash para acceder a datos o para la seguridad. Un valor hash (o simplemente hash), también llamado un resumen del mensaje, es un número generado a partir de una cadena de texto. El hash es sustancialmente menor que el propio texto, y se genera por una fórmula de tal manera que es muy poco probable que algún otro texto producirá el mismo valor hash.}}

\newglossaryentry{Firewall}{name={Firewall},
description={es software o hardware que ayuda a evitar que los hackers y algunos tipos de malware de llegar a su PC a través de una red o de Internet. Para ello, el control de la información que viene de Internet o una red y, a continuación, ya sea bloqueando o permitiendo que pase a través de su PC.}}
%Fin de glosario

\newglossaryentry{STRIDE}{name={STRIDE},
description={es un sistema desarrollado por Microsoft para pensar acerca de las amenazas de seguridad informática. Proporciona una regla mnemotécnica para amenazas de seguridad en seis categorías: Spoofing of user identity, Tampering, Repudiation, Information disclosure (privacy breach or data leak), Denial of service (D.o.S), Elevation of privilege}}

\newglossaryentry{UNIX}{name={UNIX},
description={es un sistema operativo portable, multitarea y multiusuario desarrollado, en principio, en 1969, por un grupo de empleados de los laboratorios Bell de AT\&T, entre los que figuran Dennis Ritchie, Ken Thompson y Douglas McIlroy}}

\newglossaryentry{Flexible}{name={Flexible},
description={es la capacidad de hacer cambios en el producto que se está desarrollando, o en la forma en que se desarrolla, hasta relativamente tarde en el desarrollo, sin ser demasiado perturbador. En consecuencia, la posterior puede hacer cambios, entre más flexible es el proceso menos perjudicial es}}

\newglossaryentry{Tamper-Resistant-Hardware}{name={Tamper Resistant Hardware},
description={son dispositivos resistentes a las manipulaciones  y no permiten ciertas ser extraídos fuera del hardware. Esto puede proporcionar una garantía muy fuerte ya que no pueden ser objeto de abuso: la única manera para manipularlos es poseer físicamente el dispositivo en particular}}

%Fin de glosario

%Acronimos
\newacronym{SDL}{SDL}{Security Development Lifecicle}
\newacronym{RSL}{RSL}{Revisión Sistemática de la Literatura}
\newacronym{XSS}{XSD}{Cross-Site Scripting}
\newacronym{SQL}{SQL}{Structured Query Language}
\newacronym{AES}{AES}{Advanced Encryption Standard}
\newacronym{URL}{URL}{Uniform Resource Locator}
\newacronym{OWASP}{OWASP}{The Open Web Application Security Project}
\newacronym{CLASP}{CLASP}{Comprehensive, Lightweight Application Security Process}
\newacronym{CVDS}{CVDS}{Ciclo de Vida de Desarrollo de Software}
\newacronym{CVTS}{CTDS}{Ciclo Tradicional de Desarrollo de Software}
\newacronym{CAS}{CAS}{Code Access Security}
\newacronym{CUV}{CUV}{Casos de Uso de Vulnerabilidades}
\newacronym{IT}{IT}{Information Technologies}
\newacronym{VC}{VC}{Vista de Conceptos}
\newacronym{VBR}{VBR}{Vista Basada en Roles}
\newacronym{VBEA}{VBEA}{Vista Basada en Evaluación de Actividades}
\newacronym{VBIA}{VBIA}{Vista Basada en la Implementación de Actividades}
\newacronym{VV}{VV}{Vista de Vulnerabilidades}
\newacronym{LV}{LV}{Léxico de Vulnerabilidades}
\newacronym{VS}{VS}{Vulnerabilidad de Seguridad}
\newacronym{FOSS}{FOSS}{Free and Open-Source Software}
\newacronym{SAMM}{SAMM}{Software Assurance Maturity Model}
\newacronym{CbyC}{CbyC}{Correctness by Construction}
\newacronym{CbyD}{CbyD}{Correctness by Debuging}
\newacronym{PSP}{PSP}{Personal Software Process}
\newacronym{TSP}{TSP}{Team Software Process}
\newacronym{UML}{UML}{Unified Modeling Language}
\newacronym{CMMI}{CMMI}{Capability Maturity Model Integration}
\newacronym{SREP}{SREP}{Security Requirements Engineering Process}
\newacronym{SIS}{SIS}{Security Information Systems}
\newacronym{SRR}{SRR}{Security Resource Repository}
\newacronym{MAGERIT}{MAGERIT}{Metodología de Análisis y Gestión de Riesgos de los Sistemas de Información}
\newacronym{SRRD}{SRRD}{Security Requirements Rationale Document}
\newacronym{STD}{STD}{Security Target Document}
\newacronym{TOE}{TOE}{Target of Evaluation}
\newacronym{ST}{ST}{Security Target}
\newacronym{PP}{PP}{Protection Profile}
\newacronym{EAL}{EAL}{Evaluation Assurance Level}
\newacronym{PO}{PO}{Product Owner}
\newacronym{UP}{UP}{Unified Process}
\newacronym{IC}{IC}{Integrated Circuit}
\newacronym{ISO}{ISO}{International Organization for Standardization}
\newacronym{CCEAL}{CCEAL}{Common Criteria Evaluation Assurance Levels}
\newacronym{CoSMo}{CoSMo}{Conceptual Security Modeling}
\newacronym{IS}{IS}{Information Systems}
\newacronym{DAC}{DAC}{Discreationary Access Control}
\newacronym{MAC}{MAC}{Mandatory Access Control}
\newacronym{RBAC}{RBAC}{Role Based Access Control}
\newacronym{OSI}{OSI}{Open Systems Interconnection Model}
\newacronym{CA}{CA}{Certification Authority}
\newacronym{SI}{SI}{Seguridad Informática}
\newacronym{TI}{TI}{Tecnologías de la Información}
\newacronym{OMG}{OMG}{Object Management Group}
\newacronym{MDA}{MDA}{Model Driven Architecture}
\newacronym{RFPs}{RFPs}{Requests of Proposals}
\newacronym{CORBA}{CORBA}{Common Object Request Broker Architecture}
\newacronym{HBCI}{HBCI}{Homebancking Computer Interface}
\newacronym{CEPS}{CEPS}{Common Electronic Purse Specifications}
\newacronym{BSIMM}{BSIMM}{Building  Security in Maturity Model}
\newacronym{OSSTMM}{OSSTMM}{Open Source Security Testing Methodology Manual}






%\newacronym{FOSS}{FOSS}{Free and Open Source Software}
%Fin de acronimos

\newcommand{\keywords}[1]{\par\addvspace\baselineskip
\noindent\keywordname\enspace\ignorespaces#1}

\begin{document}

\mainmatter  
\title{Estado del arte}
\titlerunning{Estado del arte}
\author{Ing. Felipe de Jesús Miramontes Romero}
\institute{Centro de Investigación en Matemáticas A.C.,\\Maestría en Ingeniería de Software,\\
Avenida Universidad 222, La Loma, 98068, Zacatecas, México.\\felipemiramontesr@gmail.com\\
\url{http://www.ingsoft.mx}}
\maketitle

\section{Introducción}
Como base fundamental de este estudio se ha realizado una revisión sistemática de la literatura acerca del desarrollo de aplicaciones en red con características enfocadas a la seguridad y las técnicas existentes para su correcta construcción, el resultado del análisis muestra que la literatura existente posee una clara tendencia hacia la generalización de los tipos de \gls{Software} pues en su mayoría las técnicas estudiadas consideran a las aplicaciones en red como \gls{Software} genérico, por otra parte y a manera de excepción existen muy pocas que tratan el tema de forma particular.

\section{Protocolo de la revisión sistemática}
El protocolo de la revisión sistemática es definido como una serie de pasos o etapas que pretenden ayudar al investigador a reunir el suficiente material de estudio sobre alguna temática de importancia, entre los principales beneficios obtenidos se encuentran el enriquecimiento de la base de conocimientos, las actividades a realizar y las decisiones tomadas a lo largo de la investigación, a continuación son mencionadas cada una de las etapas que lo conforman.

\begin{enumerate}
\item Planificación.\\ 

En esta etapa es necesario establecer los objetivos y el rumbo que debe tomar la investigación. A continuación son mencionadas las actividades que se deben realizar.\\

	\begin{enumerate}
		\item Realizar una adecuada elección del tema de investigación.
		\item Seleccionar cadenas y fuentes de búsqueda.
		\item Establecer criterios para la elección de estudios primarios y secundarios.\\
		
	\end{enumerate}
\item Revisión o ejecución.\\

En esta etapa es necesario ejecutar ciertas actividades que garantizan un correcto desarrollo. Las actividades se mencionan a continuación.

	\begin{enumerate}
		\item Ejecutar las búsquedas de información.
		\item Evaluar la calidad de la información.
		\item Revisar cada uno de los estudios seleccionados.
		\item Extraer la información relevante y necesaria.
		\item Documentar cada una de las interacciones para llevar un registro histórico que permita controlar el rumbo de la investigación.\\
		
	\end{enumerate}
\item Publicación.\\

En esta etapa es necesario exponer de manera formal los resultados de nuestra investigación por medio de la redacción de un documento formal, en este caso la tesis, en la cual se deben incluir cada uno de los cálculos estadísticos y numéricos realizados.
 
\end{enumerate}

\section{Revisión sistemática para la tesis}
El objetivo principal de la \gls{RSL} en la ingeniería de software es proporcionar los medios para suministrar la evidencia de mayor calidad y formalidad de la investigación actual e integrarla con la experiencia practica para lograr la mejor toma de decisiones en relación con el desarrollo y el mantenimiento de \gls{Software}. La revisión sistemática sera realizada para resolver una problemática concreta, en este caso en particular, la temática que se pretende abordar es la construcción de una técnica para el desarrollo de aplicaciones seguras en red (Aplicaciones Web). La revisión sistemática se apega a las necesidades esta investigación pues sus principales metas son identificar, evaluar y interpretar la mayoría de los recursos literarios disponibles acerca de un tema, cuestión, tópico, área o fenómeno de interés, mismas que sirvieran como base para el desarrollo de la tesis y que además mostraran a los interesados el estado actual de la linea de investigación (Estado del Arte). Debido a la formalidad que la revisión sistemática posee y debido a su reconocimiento como protocolo en comparación con métodos tradicionales se espera que la construcción y redacción del estado del arte sea realizado en tiempo y forma.
\\\\
Para la recolección del material de estudio utilizando la revisión sistemática se formularon las siguientes preguntas de investigación:

\begin{enumerate}
	\item ¿Cuáles técnicas son usadas para desarrollar aplicaciones en red seguras?	
	\item ¿Cuáles son los principales beneficios que las técnicas para el desarrollo de aplicaciones en red seguras aportan a los involucrados en el desarrollo del producto? 
	\item ¿Cuáles son las principales deficiencias existentes en las técnicas para el desarrollo de aplicaciones en red seguras?
	
\end{enumerate}

Una vez definidas las preguntas de investigación fueron creadas cadenas de palabras unidas por medio de operadores binarios y fueron usadas para optimizar la búsqueda e identificación del material de investigación apropiado. Dichas cadenas fueron formadas por medio de palabras claves ligadas a la temática principal y las cuales a continuación son mencionadas:

\begin{enumerate}
	\item Técnicas AND Desarrollo AND Aplicaciones en Red AND Seguras
	\item Deficiencias OR Problemas AND Técnica AND Desarrollo AND Aplicaciones en Red AND Seguras
	\item Aseguramiento AND Seguridad AND Aplicaciones en red
	\item Desarrollo de Software AND Seguro
\end{enumerate}

Una vez ejecutada la búsqueda de las cadenas anteriormente mencionadas los estudios obtenidos fueron analizados y evaluados para su inclusión en la literatura de estudio por medio de cumplimento de los siguientes criterios de aceptación:

\begin{enumerate}
	\item ¿La referencia se encuentra publicada en idioma ingles o español?
	\item ¿La referencia explícitamente ha sido publicada en años posteriores al 2009 o se ha actualizado en años posteriores al 2009?
	\item ¿La referencia proporciona datos fiables y comprobables?
	\item ¿La referencia explícitamente discute algún aspecto sobre técnicas para el desarrollo de aplicaciones en red seguras?
\end{enumerate}

\section{Literatura incluida en el estado del arte}
La literatura que se ha incluido en el estado del arte ha cumplido las paramétricas establecidas a lo largo del desarrollo del proceso definido como revisión sistemática, se han incluido trabajos realizados por entidades académicas y de investigación así como trabajos desarrollados por entidades privadas, empresas, organizaciones gubernamentales y comunidades abiertas, todo esto con el fin de establecer una base confiable para el desarrollo de la nueva técnica para el desarrollo de aplicaciones seguras en red. A continuación son presentadas las técnicas recabadas en el estudio.
 


\printnoidxglossaries     
         
\bibliographystyle{ieeetr}
\bibliography{bibliography}


\end{document}
